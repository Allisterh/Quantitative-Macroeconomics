% !TeX encoding = UTF-8
% !TeX spellcheck = en_US
\documentclass[a4paper]{scrartcl}
\usepackage[T1]{fontenc}
%\usepackage[utf8]{inputenc}
\usepackage[english]{babel} \usepackage[bottom=2.5cm,top=2.0cm,left=2.0cm,right=2.0cm]{geometry}
\usepackage{amssymb,amsmath,amsfonts}
\usepackage{lmodern}
\usepackage{graphicx}
\usepackage{csquotes}
\usepackage[usenames,dvipsnames]{xcolor}
\definecolor{mygreen}{rgb}{0,0.4,0}
\definecolor{mygray}{rgb}{0.2,0.2,0.2}
\usepackage[numbered,framed]{matlab-prettifier}
\usepackage[backend=biber,style=authoryear]{biblatex}
\addbibresource{template_exam_biblio.bib}

\begin{document}
\title{Quantitative Macroeconomics\\Midterm Exam}
\author{Willi Mutschler\\Student ID: 123\\willi@mutschler.eu}
\date{Version: \today}
\maketitle\thispagestyle{empty}

\newpage
\tableofcontents\thispagestyle{empty}\newpage \setcounter{page}{1}

\section{Exercise 1}\label{sec:introduction}
This is a \LaTeX template you might find useful to hand in your exam.

\section{Tables}
Table \ref{tbl:1} is an example of a table.
\begin{table}[h!]
  \centering
  \begin{tabular}{|l|c|r|}
  \hline
  \multicolumn{3}{|c|}{Country List} \\
  \hline
  Country Name or Area Name& ISO ALPHA 2 Code &ISO ALPHA 3 \\ \hline
  Albania &AL & ALB \\
  Algeria &DZ & DZA \\
  American Samoa & AS & ASM \\
  Angola & AO & AGO \\
  \hline
  \end{tabular}
  \caption{This is the caption for the example table.} \label{tbl:1}
\end{table}

\section{Figures}
Table \ref{fig:1} is an example of a figure, remove the draft option to actually print it.
\begin{figure}[t!]\centering
  \includegraphics[draft,width=0.5\textwidth]{mycoolplot.pdf}
  \caption{This is the caption for the example figure.}
  \label{fig:1}
\end{figure}

\section{Math}
\begin{equation}
e^{\pi i} + 1 = 0\label{eq:euler}
\end{equation}

The beautiful equation \eqref{eq:euler} is known as the Euler equation. We can also align equations:
\begin{align*}
x&=y           &  w &=z              &  a&=b+c\\
2x&=-y & 3w&=\frac{1}{2}z & -4 + 5x&=2+y & w+2&=-1+w &
a&=b\\
ab&=cb
\end{align*}
Inline math works like this $x_t=A x_{t-1}+ \varepsilon_t$ where $\varepsilon_t \overset{iid}\sim \underbrace{N(\underset{3 \times 1}{\boldsymbol{0}},\boldsymbol{\Sigma})}_{\text{normally distributed}}$.
We can also break long lines:
\begin{multline}
p(x) = 3x^6 + 14x^5y + 590x^4y^2 + 19x^3y^3
\\
- 12x^2y^4 - 12xy^5 + 2y^6 - a^3b^3 \label{eq:long}
\end{multline}
Equation \eqref{eq:long} is a long equation.
Or group and center lines:
\begin{gather*}
y_t = C x_{t}\\
x_t = A x_{t-1} + Bu_t
\end{gather*}

\section{Displaying code}
Use \texttt{lstlisting} to display code directly:
\begin{lstlisting}[
style = Matlab-editor, basicstyle = \mlttfamily,
]
x = reshape(eye(3,3),3*3,1);
dlyap(A,RHS);
\end{lstlisting}
or load it from a file with \texttt{lstinputlisting}
\lstinputlisting[
        style = Matlab-editor,
        basicstyle = \mlttfamily,
  ]{template_matlab_example.m}
Note that the pretty formatting for MATLAB is achieved by loading the package \texttt{matlab-prettifier}.

\section{Citations}\label{sec:citations}
Examples how to do citations:
\begin{itemize}
  \item \textcite{Sims_1980_MacroeconomicsReality} shows that SVAR models can be used to study the transmission channel of monetary policy shocks.
  \item The book \textcite{Herbst.Schorfheide_2016_BayesianEstimationDSGE} emphasizes that DSGE models are usually estimated by Bayesian methods.
  \item Identification plays an important role in Quantitative Macroeconomics \parencite{Kilian_2013_StructuralVectorAutoregressions,Mutschler_2022_QuantitativeMacroeconomics}.
\end{itemize}
Now let's print the bibliography.
\printbibliography

\end{document}